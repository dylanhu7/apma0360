\documentclass[plain]{pset}
\usepackage{dylan}

\title{Homework 5}
\author{Dylan Hu}
\prof{Professor Zhuolun Yang}
\course{APMA 0360 --- Partial Differential Equations}
\date{March 17, 2024}

\begin{document}

\begin{multicols}{2}
    \raggedcolumns{}
    \maketitle
    \columnbreak{}
    \tableofcontents
\end{multicols}

\setlength{\parskip}{1em}
\setlength{\parindent}{0pt}

\pagebreak

\begin{problem}
\[\]
\vspace{-4em}
\begin{enumerate}[a.]
    \item Verify \(\cos(nx)\)and \(\cos(mx)\) are orthogonal on \((0, \pi)\) when \(n \neq m\) and \(n, m \in \mathbb{N}\). That is
          \[
              \int_0^\pi \cos(nx)\cos(mx) \dd x= 0.
          \]
    \item Verify \(\sin(nx)\) and \(\sin(mx)\) are orthogonal on \((-\pi, \pi)\) when \(n, m \in \mathbb{N}\). That is
          \[
              \int_{-\pi}^\pi \sin(nx)\cos(mx) \dd x= 0.
          \]
\end{enumerate}
\end{problem}
\begin{solution}
    \[\]
    \vspace{-4em}
    \begin{enumerate}[a.]
        \item
              \begin{align*}
                  \int_0^\pi \cos(nx)\cos(mx) \dd x & = \frac{1}{2}\int_0^\pi \left(\cos((n+m)x) + \cos((n-m)x)\right) \dd x                 \\
                                                    & = \frac{1}{2}\left[\frac{1}{n+m}\sin((n+m)x) + \frac{1}{n-m}\sin((n-m)x) \right]_0^\pi \\
                                                    & = \frac{1}{2}\left[\frac{1}{n+m}\sin((n+m)\pi) + \frac{1}{n-m}\sin((n-m)\pi) \right]   \\
                                                    & = 0
              \end{align*}
              We have used the fact that \(\sin(m\pi) = \sin(n\pi)= 0\) for \(m, n \in \mathbb{N}\) and that \(n \neq m \implies n \pm m \neq 0\).
        \item
              \begin{align*}
                  \int_{-\pi}^\pi \sin(nx)\cos(mx) \dd x & = \frac{1}{2} \int_{-\pi}^\pi \left(\sin((n+m)x) + \sin((n-m)x)\right) \dd x                                                 \\
                                                         & = \frac{1}{2} \int_{-\pi}^\pi \left(\sin((n+m)x) \right) \dd x + \frac{1}{2} \int_{-\pi}^\pi \left(\sin((n-m)x)\right) \dd x \\
                                                         & = 0 + 0 = 0
              \end{align*}
              We use the fact that \(\sin\) is an odd function and \((-\pi, \pi)\) is symmetric, so \(\int_{-\pi}^\pi \sin((n+m)x) \dd x = 0\) and \(\int_{-\pi}^\pi \sin((n-m)x) \dd x = 0\).
    \end{enumerate}
\end{solution}

\pagebreak

\begin{problem}
Solve
\[
    \begin{cases}
        \begin{aligned}
            u_t - u_{xx}        & = 0,   &  & 0 < x < \pi, & t > 0, \\
            u_x(0, t) = u_x(\pi, t) & = 0,   &  & t > 0,                \\
            u(x, 0)             & = x^2, &  & 0 < x < \pi.
        \end{aligned}
    \end{cases}
\]
explicitly.
\end{problem}

\begin{solution}
    \[u(x, t) = X(x)T(t)\]
    Using the heat equation, we have
    \[
        X(x)T'(t) - X''(x)T(t) = 0 \implies \frac{T'(t)}{T(t)} = \frac{X''(x)}{X(x)} = \lambda
    \]
    As in HW4, we have the following eigenvalue problem
    \[
        \begin{cases}
            \begin{aligned}
                X''(x) + \lambda X(x) & = 0, &  & 0 < x < \pi, \\
                X'(0) = X'(\pi)       & = 0.
            \end{aligned}
        \end{cases}
    \]
    We recall that in the case of \(\lambda = 0\), we have \(X(x) = a_0\) and in the case of \(\lambda < 0\), we have \(X(x) = a\cos(\omega x) + b\sin(\omega x)\) where \(\lambda = -\omega^2\) and \(\omega \in \mathbb{R}^+\). Using the boundary conditions:
    \[
        \begin{cases}
            \begin{aligned}
                X'(0) = 0 & \implies -a\omega\sin(0) + b\omega\cos(0) = 0 \implies b\omega = 0 \implies b = 0, \\
                X'(\pi) = 0 & \implies -a\omega\sin(\omega\pi) = 0 \implies \sin(\omega\pi) = 0 \implies \omega = n, n \in \mathbb{N}.
            \end{aligned}
        \end{cases}
    \]
    We now turn to the ODE in \(T(t)\):
    \[
        \frac{T'(t)}{T(t)} = \lambda \implies T'(t) = \lambda T(t) = -n^2T(t) \implies T(t) = Ce^{-n^2t}
    \]
\end{solution}


\end{document}
