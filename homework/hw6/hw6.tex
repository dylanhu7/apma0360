\documentclass[plain]{pset}
\usepackage{dylan}

\newcommand{\p}{\partial}
% \newcommand{\dd}{\mathop{}\!\mathrm{d}}
\newcommand{\dv}[3][]{\frac{\dd^{#1} #2}{\dd #3^{#1}}}
\newcommand{\pdv}[3][]{\frac{\p^{#1} #2}{\p #3^{#1}}}
\newcommand{\del}{\nabla}

\title{Homework 6}
\author{Dylan Hu}
\prof{Professor Zhuolun Yang}
\course{APMA 0360 --- Partial Differential Equations}
\date{April 7, 2024}

\begin{document}

\begin{multicols}{2}
    \raggedcolumns{}
    \maketitle
    \columnbreak{}
    \tableofcontents
\end{multicols}

\setlength{\parskip}{1em}
\setlength{\parindent}{0pt}

\pagebreak

\begin{problem}
Assume that \(H(x, t),\ I(x,t)\) satisfy
\begin{align*}
    H_t & = -bHI,                     \\
    I_t & = bHI - \gamma I + DI_{xx},
\end{align*}
Define
\[\tau \coloneqq \gamma t, \quad h \coloneqq \frac{H}{N}, \quad i \coloneqq \frac{I}{N}, \quad R_0 \coloneqq \frac{bN}{\gamma}, \quad d \coloneqq \frac{D}{\gamma},\]
where \(b, \gamma, N, D\) are some positive constants. Show that \(h(x,\tau),\ i(x, \tau)\) satisfy the system
\begin{align*}
    h_\tau & = -R_0hi,              \\
    i_\tau & = R_0hi - i + di_{xx}.
\end{align*}
\textit{Hint: chain rule.}
\end{problem}
\begin{solution}
    \begin{align*}
        h_\tau & = \pdv{h}{\tau}                                                                                        \\
               & = \pdv{h}{t}\pdv{t}{\tau}                                                                              \\
               & = \left(\pdv{}{t}\left(\frac{H}{N}\right)\right)\frac{1}{\gamma}                                       \\
               & = \frac{1}{N}\pdv{H}{t}\frac{1}{\gamma}                                                                \\
               & = \frac{1}{N}\left(-bHI\right)\frac{1}{\gamma}                                                         \\
               & = -\frac{bHI}{N}\frac{R_0}{bN}                                                                         \\
               & = -R_0 \frac{H}{N}\frac{I}{N}                                                                          \\
               & = -R_0 hi                                                                                              \\
               &                                                                                                        \\
        i_\tau & = \pdv{i}{\tau}                                                                                        \\
               & = \pdv{i}{t}\pdv{t}{\tau}                                                                              \\
               & = \left(\pdv{}{t}\left(\frac{I}{N}\right)\right)\frac{1}{\gamma}                                       \\
               & = \frac{1}{N}\pdv{I}{t}\frac{1}{\gamma}                                                                \\
               & = \frac{1}{N}\left(bHI - \gamma I + DI_{xx}\right)\frac{1}{\gamma}                                     \\
               & = \frac{bHI}{N}\frac{R_0}{bN} - \frac{\gamma I}{N}\frac{1}{\gamma} + \frac{DI_{xx}}{N}\frac{1}{\gamma} \\
               & = R_0 hi - \frac{I}{N} + \frac{DI_{xx}}{N\gamma}                                                       \\
               & = R_0 hi - i + d\frac{I_{xx}}{N}                                                                       \\
               & = R_0 hi - i + di_{xx} \quad \qquad \left(i_{xx} = \frac{I_{xx}}{N}\right)
    \end{align*}
\end{solution}

\pagebreak

\begin{problem}
For \((x, y) \neq (0, 0)\), compute the Laplacian of
\[u(x, y) = \ln\left(x^2 + y^2\right)\]
and conclude whether it satisfies the Laplace equation for \((x, y) \neq (0, 0)\).
\end{problem}
\begin{solution}
\end{solution}

\pagebreak

\begin{problem}
Use separation of variables to solve
\[
    \begin{cases}
        \begin{aligned}
            u_{xx} + u_{yy} & = 0,             &  & 0 < x, y < \pi, \\
            u_y(x, 0)       & = 0,             &  & 0 < x < \pi,    \\
            u_y(x, \pi)     & = 0,             &  & 0 < x < \pi,    \\
            u(0, y)         & = 0,             &  & 0 < y < \pi,    \\
            u(\pi, y)       & = 1 + 3\cos(2y), &  & 0 < y < \pi.
        \end{aligned}
    \end{cases}
\]
\end{problem}
\begin{solution}

\end{solution}

\pagebreak

\begin{problem}
Use separation of variables to solve
\[
    \begin{cases}
        \begin{aligned}
            u_{xx} + u_{yy} & = 0,   &  & 0 < x, y < \pi, \\
            u(x, 0)         & = 0,   &  & 0 < x < \pi,    \\
            u(x, \pi)       & = 100, &  & 0 < x < \pi,    \\
            u(0, y)         & = 0,   &  & 0 < y < \pi,    \\
            u(\pi, y)       & = 100, &  & 0 < y < \pi.
        \end{aligned}
    \end{cases}
\]
\end{problem}
\begin{solution}

\end{solution}

\end{document}
