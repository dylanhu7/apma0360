\documentclass[plain]{pset}
\usepackage{dylanhu}

\title{Homework 4}
\author{Dylan Hu}
\prof{Professor Zhuolun Yang}
\course{APMA 0360 --- Partial Differential Equations}
\date{March 10, 2024}

\begin{document}

\maketitle

\pagebreak

\begin{problem}
Verify the superposition principle: if \(u_1\) and \(u_2\) are two solutions to
\[
    \begin{cases}
        \begin{aligned}
            u_t - u_{xx}        & = 0, & 0 & < x < \pi, & t > 0, \\
            u(0, t) = u(\pi, t) & = 0, & t & > 0                 \\
        \end{aligned}
    \end{cases}
\]
then so is \(a_1u_1 + a_2u_2\) for any constants \(a_1, a_2\). Is it still true if the boundary condition is \(u(0, t) = u(\pi, t) = 1\)?
\end{problem}
\begin{solution}
    We can verify the superposition principle by plugging in \(u_1\) and \(u_2\) into the heat equation and adding them together. We have
    \[
        \begin{aligned}
            (a_1u_1 + a_2u_2)_t - (a_1u_1 + a_2u_2)_{xx} & = a_1u_{1t} + a_2u_{2t} - a_1u_{1xx} - a_2u_{2xx}   \\
                                                         & = a_1u_{1xx} + a_2u_{2xx} - a_1u_{1xx} - a_2u_{2xx} \\
                                                         & = 0
        \end{aligned}
    \]
    and
    \[
        \begin{aligned}
            (a_1u_1 + a_2u_2)(0, t) & = a_1u_1(0, t) + a_2u_2(0, t) \\
                                    & = a_1(0) + a_2(0)             \\
                                    & = 0
        \end{aligned}
    \]
    and
    \[
        \begin{aligned}
            (a_1u_1 + a_2u_2)(\pi, t) & = a_1u_1(\pi, t) + a_2u_2(\pi, t) \\
                                      & = a_1(0) + a_2(0)                 \\
                                      & = 0
        \end{aligned}
    \]
    so \(a_1u_1 + a_2u_2\) is also a solution.

    If the boundary condition is \(u(0, t) = u(\pi, t) = 1\), then we have
    \[
        \begin{aligned}
            (a_1u_1 + a_2u_2)(0, t) & = a_1u_1(0, t) + a_2u_2(0, t) \\
                                    & = a_1(1) + a_2(1)             \\
                                    & = a_1 + a_2
        \end{aligned}
    \]
    \(a_1 + a_2\) is not 1 for all \(a_1, a_2\), so the superposition principle does not hold.
\end{solution}

\pagebreak

\begin{problem}
Use separation of variables to solve
\[
    \begin{cases}
        \begin{aligned}
            u_t - u_{xx}            & = 0,             & 0 & < x < \pi, & t > 0, \\
            u_x(0, t) = u_x(\pi, t) & = 0,             & t & > 0,                \\
            u(x, 0)                 & = 1 + 2\cos(3x), & 0 & < x < \pi           \\
        \end{aligned}
    \end{cases}
\]
explicitly. You don't need to go through all 3 cases.
\end{problem}
\begin{solution}
    We look for a solution of the form \(u(x, t) = X(x)T(t)\). Plugging this into the heat equation, we have
    \[X(x)T'(t) - X''(x)T(t)  = 0\]
    \[\frac{T'(t)}{T(t)}      = \frac{X''(x)}{X(x)} = \lambda\]
    The boundary conditions give us
    \[u_x(0, t) = X'(0)T(t) = 0 \implies X'(0) = 0\]
    \[u_x(\pi, t) = X'(\pi)T(t) = 0 \implies X'(\pi) = 0\]
    We now have a system of ODEs for \(X(x)\):
    \[
        \begin{cases}
            \begin{aligned}
                X''(x) - \lambda X(x) & = 0, & 0 & < x < \pi, \\
                X'(0) = X'(\pi)       & = 0, &   &            \\
            \end{aligned}
        \end{cases}
    \]
    Let us define
    \[V(x) = \begin{bmatrix}
            X'(x) \\
            X(x)  \\
        \end{bmatrix}\]
    Then we have
    \[V'(x) = \begin{bmatrix}
            X''(x) \\
            X'(x)  \\
        \end{bmatrix} = \begin{bmatrix}
            0 & \lambda \\
            1 & 0       \\
        \end{bmatrix}\begin{bmatrix}
            X'(x) \\
            X(x)  \\
        \end{bmatrix}
    \]
    So we have
    \[V'(x) = A\,V(x), \quad A = \begin{bmatrix}
            0 & \lambda \\
            1 & 0       \\
        \end{bmatrix}\]
    We can find the eigenvalues and eigenvectors of \(A\):
    \[\det(A - \Lambda I) = \det\begin{bmatrix}
            -\Lambda & \lambda  \\
            1        & -\Lambda \\
        \end{bmatrix} = \Lambda^2 - \lambda = 0\]
    So the eigenvalues are \(\pm\sqrt{\lambda}\).

    \textbf{Case \(\mathbf{\lambda = 0}\)}. Then \(\lambda_1 = \lambda_2 = 0\).
    \[A - \lambda_1I = \begin{bmatrix}
            0 & 0 \\
            1 & 0 \\
        \end{bmatrix}
    \]
    gives us the eigenvector \(\begin{bmatrix}
        0 \\
        1 \\
    \end{bmatrix}\). We can find a generalized eigenvector by solving
    \[
        \begin{bmatrix}
            0 & 0 \\
            1 & 0 \\
        \end{bmatrix}\begin{bmatrix}
            v_1 \\
            v_2 \\
        \end{bmatrix} = \begin{bmatrix}
            0 \\
            1 \\
        \end{bmatrix}
    \]
    which gives us \(v_1 = 1\). So the generalized eigenvector is \(\begin{bmatrix}
        1 \\
        0 \\
    \end{bmatrix}\).
    We have the solution
    \[V(x) = a_1 e^{0x}\begin{bmatrix}
            0 \\
            1 \\
        \end{bmatrix} + a_2 e^{0x}\left(x\begin{bmatrix}
                1 \\
                0 \\
            \end{bmatrix} + \begin{bmatrix}
                1 \\
                0 \\
            \end{bmatrix}\right)\]
    which gives us
    \[X(x) = a_1 + a_2x\]
    We have
    \[X'(x) = a_2\]
    The boundary conditions give us
    \[X'(0) = X'(\pi) = a_2 = 0 \implies a_2 = 0\]
    So the solution is
    \[X(x) = a\] for any constant \(a\).

    \textbf{Case \(\mathbf{\lambda < 0}\)}. Let \(\lambda = -\omega^2\), where \(\omega > 0\). Then the eigenvalues are \(\pm i\omega\).
    This gives the solution
    \[X(x) = a_1\cos(\omega x) + a_2\sin(\omega x)\]
    for any constants \(a_1, a_2\).
    We have
    \[X'(x) = -a_1\omega\sin(\omega x) + a_2\omega\cos(\omega x)\]
    and the boundary conditions give us
    \[X'(0) = -a_1\omega\sin(0) + a_2\omega\cos(0) = 0 \implies a_2 = 0\]
    \[X'(\pi) = -a_1\omega\sin(\omega\pi) = 0 \implies \omega = n\]
    where \(n \in \mathbb{Z}^+\). So \(X(x) = a_1\cos(nx)\) and \(\lambda = -n^2\).
    We now turn to the ODE for \(T(t)\):
    \[\frac{T'(t)}{T(t)} = \lambda = -n^2\]
    \[T'(t) = -n^2T(t)\]
    \[T(t) = Ce^{-n^2t}\]
    where \(C\) is a constant.
    A solution is then
    \[u(x, t) = X(x)T(t) = Ce^{-n^2t}\cos(nx)\]
    By the principle of superposition, the general solution is
    \[u(x, t) = a + \sum_{n=1}^\infty a_n e^{-n^2t}\cos(nx)\]
    where \(a_n\) are constants. We can find the constants by using the initial condition:
    \[u(x, 0) = a + \sum_{n=1}^\infty a_n\cos(nx) = 1 + 2\cos(3x)\]
    We have \(a = 1\), \(a_3 = 2\), and \(a_n = 0\) for all other \(n\). So the solution is
    \[u(x, t) = 1 + 2e^{-9t}\cos(3x)\]
\end{solution}

\pagebreak

\begin{problem}
Consider the following heat-like equation:
\[
    \begin{cases}
        \begin{aligned}
            tu_t - u_{xx}             & = 0, & 0 & < x < \pi, & t > 0, \\
            u(0, t)       = u(\pi, t) & = 0, & t & > 0,                \\
        \end{aligned}
    \end{cases}
\]
Use separation of variables to write the solution as an infinite series. You don't need to go through all 3 cases.
\end{problem}
\begin{solution}
    Let \(u(x, t) = X(x)T(t)\). Then
    \[tX(x)T'(t) - X''(x)T(t) = 0\]
    \[\frac{tT'(t)}{T(t)} = \frac{X''(x)}{X(x)} = \lambda\]
    The boundary conditions give us
    \[u(0, t) = X(0)T(t) = 0 \implies X(0) = 0\]
    \[u(\pi, t) = X(\pi)T(t) = 0 \implies X(\pi) = 0\]
    We now have a system of ODEs for \(X(x)\):
    \[
        \begin{cases}
            \begin{aligned}
                X''(x) - \lambda X(x) & = 0, & 0 & < x < \pi, \\
                X(0) = X(\pi)         & = 0, &   &            \\
            \end{aligned}
        \end{cases}
    \]
    Like in the previous problem, we arrive at the solution
    \[X(x) = a_1\cos(\omega x) + a_2\sin(\omega x)\]
    where \(\lambda = -\omega^2\) and \(\omega > 0\).
    Using the boundary conditions, we have
    \[X(0) = a_1\cos(0) + a_2\sin(0) = a_1 = 0\]
    \[X(\pi) = a_2\sin(\omega\pi) = 0 \implies \omega = n\]
    where \(n \in \mathbb{Z}^+\). So \(X(x) = a\sin(nx)\) and \(\lambda = -n^2\).

    We now turn to the ODE for \(T(t)\):
    \[\frac{tT'(t)}{T(t)} = \lambda = -n^2\]
    \[tT'(t) = -n^2T(t)\]
    \[\frac{T'(t)}{T(t)} = -\frac{n^2}{t}\]
    \[T'(t) = -\frac{n^2}{t}T(t)\]
    \[T'(t) + \frac{n^2}{t}T(t) = 0\]
    This is a first-order linear ODE with variable coefficients. We can solve it using an integrating factor:
    \[\mu(t) = e^{\int\frac{n^2}{t}\,dt} = e^{n^2\ln(t)} = e^{\ln(t^{n^2})} = t^{n^2}\]
    \[T'(t)t^{n^2} + \frac{n^2}{t}T(t)t^{n^2} = 0\]
    \[\left(T(t)t^{n^2}\right)' = 0\]
    \[T(t)t^{n^2} = C\]
    \[T(t) = Ct^{-n^2}\]
    where \(C\) is a constant.

    A solution is then
    \[u(x, t) = X(x)T(t) = C\sin(nx)t^{-n^2}\]

    By the principle of superposition, the general solution is
    \[u(x, t) = \sum_{n=1}^\infty a_n\sin(nx)t^{-n^2}\]
    where \(a_n\) are constants.


\end{solution}

\pagebreak

\begin{problem}
Consider the heat equation with mixed boundary conditions
\[
    \begin{cases}
        \begin{aligned}
            u_t - u_{xx} & = 0, & 0 & < x < \pi, & t > 0, \\
            u_x(0, t)    & = 0, & t & > 0,                \\
            u(\pi, t)    & = 0, & t & > 0.                \\
        \end{aligned}
    \end{cases}
\]
Use separation of variables and go through all 3 cases to write the solution as an infinite series.
\end{problem}
\begin{solution}
    As before, we let \(u(x, t) = X(x)T(t)\). Then
    \[\frac{T'(t)}{T(t)} = \frac{X''(x)}{X(x)} = \lambda\]
    The boundary conditions give us
    \[u_x(0, t) = X'(0)T(t) = 0 \implies X'(0) = 0\]
    \[u(\pi, t) = X(\pi)T(t) = 0 \implies X(\pi) = 0\]
    We now have a system of ODEs for \(X(x)\):
    \[
        \begin{cases}
            \begin{aligned}
                X''(x) - \lambda X(x) & = 0, & 0 & < x < \pi, \\
                X'(0)                 & = 0  &   &            \\
                X(\pi)                & = 0  &   &            \\
            \end{aligned}
        \end{cases}
    \]
    As before, the eigenvalues are \(\pm\sqrt{\lambda}\).

    \textbf{Case \(\mathbf{\lambda > 0}\)}. Let \(\lambda = \omega^2\), where \(\omega > 0\). Then \(\lambda_1 = \omega^2\) and \(\lambda_2 = -\omega^2\).
    \[A - \lambda_1I = \begin{bmatrix}
            -\omega & \omega^2 \\
            1       & -\omega  \\
        \end{bmatrix}\]
    so an eigenvector corresponding to \(\lambda_1\) is \(\begin{bmatrix}
        \omega \\
        1      \\
    \end{bmatrix}\).
    \[A - \lambda_2I = \begin{bmatrix}
            \omega & \omega^2 \\
            1      & \omega   \\
        \end{bmatrix}\]
    so an eigenvector corresponding to \(\lambda_2\) is \(\begin{bmatrix}
        -\omega \\
        1       \\
    \end{bmatrix}\).
    We have the solution
    \[V(x) = a_1 e^{\omega x}\begin{bmatrix}
            \omega \\
            1      \\
        \end{bmatrix} + a_2 e^{-\omega x}\begin{bmatrix}
            -\omega \\
            1       \\
        \end{bmatrix}\]
    which gives us
    \[X(x) = a_1 e^{\omega x} + a_2 e^{-\omega x}\] for any constants \(a_1, a_2\).
    We have
    \[X'(x) = a_1\omega e^{\omega x} - a_2\omega e^{-\omega x}\]
    The boundary conditions give us
    \[X'(0) = a_1\omega - a_2\omega = 0 \implies a_1 = a_2\]
    \[X(\pi) = a_1 e^{\omega\pi} + a_2 e^{-\omega\pi} = a_1 e^{\omega\pi} + a_1 e^{-\omega\pi} = a_1(e^{\omega\pi} + e^{-\omega\pi}) = 0 \implies a_1 = a_2 = 0\]
    So, we have no solution for \(\lambda > 0\).

    \textbf{Case \(\mathbf{\lambda = 0}\)}. Then \(\lambda_1 = \lambda_2 = 0\).
    As before, we have the general solution
    \[X(x) = a_1 + a_2x\]
    We have
    \[X'(x) = a_2\]
    The boundary conditions give us
    \[X'(0) = a_2 = 0 \implies X(x) = a_1\]
    \[X(\pi) = a_1 = 0 \implies a_1 = 0\]
    So, we have no solution for \(\lambda = 0\).

    \textbf{Case \(\mathbf{\lambda < 0}\)}.
    As before, we have the general solution
    \[X(x) = a_1\cos(\omega x) + a_2\sin(\omega x)\]
    for any constants \(a_1, a_2\).
    We have
    \[X'(x) = -a_1\omega\sin(\omega x) + a_2\omega\cos(\omega x)\]
    and the boundary conditions give us
    \[X'(0) = -a_1\omega\sin(0) + a_2\omega\cos(0) = 0 \implies a_2 = 0\]
    \[X(\pi) = a_1\cos(\omega\pi) = 0 \implies \omega = \frac{n}{2}\]
    where \(n \in \mathbb{Z}^+\). So \(X(x) = a_1\cos\left(\frac{n}{2}x\right)\) and \(\lambda = -\frac{n^2}{4}\).

    We now turn to the ODE for \(T(t)\):
    \[\frac{T'(t)}{T(t)} = \lambda = -\frac{n^2}{4}\]
    \[T'(t) = -\frac{n^2}{4}T(t)\]
    \[T(t) = Ce^{-\frac{n^2}{4}t}\]
    where \(C\) is a constant.

    A solution is then
    \[u(x, t) = X(x)T(t) = Ce^{-\frac{n^2}{4}t}\cos\left(\frac{n}{2}x\right)\]

    By the principle of superposition, the general solution is
    \[u(x, t) = \sum_{n=1}^\infty a_n e^{-\frac{n^2}{4}t}\cos\left(\frac{n}{2}x\right)\]
    where \(a_n\) are constants.

\end{solution}

\end{document}
