\documentclass[plain]{pset}
\usepackage{dylanhu}

\DeclareMathOperator{\sinc}{sinc}

\title{Final Exam}
\author{Dylan Hu}
\prof{Professor Zhuolun Yang}
\course{APMA 0360 --- Partial Differential Equations}
\date{May 16, 2024}

\begin{document}

\maketitle

\pagebreak

\begin{problem}
Let \(u(x, t)\) be the solution of the heat equation on the real line
\[
    \begin{cases}
        \begin{aligned}
            u_t     & = u_{xx}, & x \in \R, &  & t > 0, \\
            u(x, 0) & = f(x),   & x \in \R,
        \end{aligned}
    \end{cases}
\]
where
\[
    f(x) = \begin{cases}
        1, & -1 \leq x \leq 1, \\
        0, & \abs{x} > 1.
    \end{cases}
\]
Compute \(\hat{u}(\kappa, t)\), the Fourier transform of \(u(x, t)\), explicitly. (You do not need to find the solution \(u(x, t)\).)
\end{problem}
\begin{solution}
    Recall that the Fourier transform of the heat equation in general is
    \[\hat{u}(\kappa, t) = \hat{u}(\kappa, 0) e^{-D\kappa^2 t}.\]
    Here, we have \(D = 1\) and \(\hat{u}(\kappa, 0) = \hat{f}(\kappa)\). So, it remains to compute \(\hat{f}(\kappa)\). We have
    \begin{align*}
        \hat{f}(\kappa) & =  \int_{-\infty}^\infty  e^{i\kappa x} f(x) \dd{x}                                               \\
                        & =  \int_{-1}^1 e^{i\kappa x} \dd{x}                                                               \\
                        & =  \frac{1}{i\kappa} \left[e^{i\kappa x} \right]_{-1}^1                                           \\
                        & =  \frac{1}{i\kappa} \left(e^{i\kappa} - e^{-i\kappa}\right)                                      \\
                        & =  \frac{1}{i\kappa} \left[ \cos(\kappa) + i\sin(\kappa) - \cos(-\kappa) - i\sin(-\kappa) \right] \\
                        & =  \frac{1}{i\kappa} \left[ \cos(\kappa) + i\sin(\kappa) - \cos(\kappa) + i\sin(\kappa) \right]   \\
                        & =  \frac{2i\sin(\kappa)}{i\kappa}                                                                 \\
                        & = \frac{2\sin(\kappa)}{\kappa}                                                                    \\
                        & =  2\sinc(\kappa).
    \end{align*}
    Thus, we have
    \[\hat{u}(\kappa, t) = 2\sinc(\kappa) e^{-\kappa^2 t}.\]
\end{solution}

\pagebreak

\begin{problem}
Solve the damped wave equation
\[
    \begin{cases}
        \begin{aligned}
            u_{tt} + 2u_t + u & = u_{xx},            & x \in \R, &  & t > 0, \\
            u(x, 0)           & = \frac{1}{1 + x^2}, & x \in \R,             \\
            u_t(x, 0)         & = 1,                 & x \in \R.
        \end{aligned}
    \end{cases}
\]
\begin{hint}
    Find the equation satisfied by \(v(x, t) = e^t u(x, t)\) first, and use the solution to that problem to find \(u(x, t)\).
\end{hint}
\end{problem}
\begin{solution}
    \begin{align*}
        v(x, t) & = e^t u(x, t),                  \\
        v_t     & = e^t u_t + e^t u,              \\
        v_{tt}  & = e^t u_{tt} + 2e^t u_t + e^t u \\
                & = e^t(u_{tt} + 2u_t + u)        \\
                & = e^t u_{xx}                    \\
                & = v_{xx}.
    \end{align*}
    We see that \(v(x, t)\) satisfies the undamped wave equation. The initial profiles are
    \[v(x, 0) = \frac{e^0}{1 + x^2} = \frac{1}{1 + x^2}, \quad v_t(x, 0) = e^0(u_t(x, 0) + u(x, 0)) = 1 + \frac{1}{1 + x^2}.\]
    Now, we have the system
    \[
        \begin{cases}
            \begin{aligned}
                v_{tt}    & = v_{xx},                & x \in \R, &  & t > 0, \\
                v(x, 0)   & = \frac{1}{1 + x^2},     & x \in \R,             \\
                v_t(x, 0) & = 1 + \frac{1}{1 + x^2}, & x \in \R.
            \end{aligned}
        \end{cases}
    \]
    which we can solve using d'Alembert's formula. Recall that the solution in general is
    \[v(x, t) = \frac{1}{2}f(x + ct) + \frac{1}{2}f(x - ct) + \frac{1}{2c} \int_{x - ct}^{x + ct} g(s) \dd{s}.\]
    Here, we have \(f(x) = \frac{1}{1 + x^2}\), \(g(x) = 1 + \frac{1}{1 + x^2}\), and \(c = 1\).
    \[v(x, t) = \frac{1}{2}\left(\frac{1}{1 + (x + t)^2} + \frac{1}{1 + (x - t)^2}\right) + \frac{1}{2} \int_{x - t}^{x + t} 1 + \frac{1}{1 + s^2} \dd{s}.\]

    We will leave the first two terms and compute the integral first.
    \begin{align*}
        \int_{x - t}^{x + t} 1 + \frac{1}{1 + s^2} \dd{s} & = \bigl[s + \arctan(s)\bigr]_{x - t}^{x + t}          \\
                                                          & = (x + t + \arctan(x + t)) - (x - t + \arctan(x - t)) \\
                                                          & = 2t + \arctan(x + t) - \arctan(x - t).
    \end{align*}
    Thus, we have
    \[v(x, t) = \frac{1}{2}\left(\frac{1}{1 + (x + t)^2} + \frac{1}{1 + (x - t)^2} + 2t + \arctan(x + t) - \arctan(x - t)\right).\]
    Finally, we have
    \[u(x, t) = e^{-t}v(x, t) = \frac{1}{2e^t}\left(\frac{1}{1 + (x + t)^2} + \frac{1}{1 + (x - t)^2} + 2t + \arctan(x + t) - \arctan(x - t)\right).\]
    If we let \(\xi = x + t\) and \(\eta = x - t\), we can write this as
    \[u(\xi, \eta) = \frac{1}{2}e^{\frac{\eta - \xi}{2}}\left(\frac{1}{1 + \xi^2} + \frac{1}{1 + \eta^2} + \xi - \eta + \arctan(\xi) - \arctan(\eta)\right).\]
\end{solution}

\pagebreak

\begin{problem}
Find the general solution of
\[
    \begin{cases}
        \begin{aligned}
            u_{xx} - 2u_x + u_{yy} & = 0, &  & 0 < x, y < \pi, \\
            u(x, 0)                & = 0, &  & 0 < x < \pi,    \\
            u(x, \pi)              & = 0, &  & 0 < x < \pi
        \end{aligned}
    \end{cases}
\]
using separation of variables.
\end{problem}
\begin{solution}
    Let \(u(x, y) = X(x)Y(y)\). Then
    \[X''Y - 2X'Y + XY'' = 0 \implies \frac{2X' - X''}{X} = \frac{Y''}{Y} = \lambda.\]
    We solve the ODE in \(y\) first, as we have the boundary conditions in \(y\).
    \[
        \begin{cases}
            \begin{aligned}
                Y''           & = \lambda Y, & 0 < y < \pi, \\
                Y(0) = Y(\pi) & = 0.
            \end{aligned}
        \end{cases}
    \]
    Recall that the general solution for the homogeneous Dirichlet boundary conditions with \(\lambda = -n^2\) is
    \[Y_n(y) = c_n \sin(ny), \quad n = 1, 2, 3, \ldots\]
    where \(c\) is a constant.

    Now, we solve the ODE in \(x\).
    \[\frac{2X' - X''}{X} = \lambda \implies X'' - 2X' + \lambda X = 0.\]
    We have the auxiliary equation
    \[r^2 - 2r + \lambda = 0 \implies r = 1 \pm \sqrt{1 - \lambda} = 1 \pm \sqrt{1 + n^2}.\]
    Since the roots are real and distinct, the general solution is
    \begin{align*}
        X_n(x) & = \tilde{a}_n e^{(1 + \sqrt{1 + n^2})x} + \tilde{b}_n e^{(1 - \sqrt{1 + n^2})x}                              \\
               & = e^x\left(\tilde{a}_n e^{\sqrt{1 + n^2}x} + \tilde{b}_n e^{-\sqrt{1 + n^2}x}\right)                         \\
               & = e^x\left(\hat{a}_n \cosh\left(\sqrt{1 + n^2}x\right) + \hat{b}_n \sinh\left(\sqrt{1 + n^2}x\right)\right).
    \end{align*}

    The general solution is then
    \begin{align*}
        u(x, y) & = \sum_{n = 1}^\infty e^x\left(\hat{a}_n \cosh\left(\sqrt{1 + n^2}x\right) + \hat{b}_n \sinh\left(\sqrt{1 + n^2}x\right)\right) c_n\sin(ny) \\
                & = e^x \sum_{n = 1}^\infty \left(a_n \cosh\left(\sqrt{1 + n^2}x\right) + b_n \sinh\left(\sqrt{1 + n^2}x\right)\right) \sin(ny).
    \end{align*}
\end{solution}

\pagebreak

\begin{problem}
Let \(u(x, t)\) be the solution of the following heat equation with mixed boundary conditions
\[
    \begin{cases}
        \begin{aligned}
            u_t         & = u_{xx} + 3x^2, &  & 0 < x < \pi, &  & t > 0, \\
            u(0, t)     & = 0,             &  & t > 0,                   \\
            u_x(\pi, t) & = 1,             &  & t > 0,                   \\
            u(x, 0)     & = \sin(x),       &  & 0 < x < \pi.
        \end{aligned}
    \end{cases}
\]
Find the limiting function \(u_\infty(x) = \lim_{t \to \infty} u(x, t)\).
\end{problem}
\begin{solution}
    We can devise functions \(w(x, t)\) and \(v(x)\) such that
    \[w(x, t) = u(x, t) - v(x, t)\]
    and
    \[
        \begin{cases}
            \begin{aligned}
                w_t         & = w_{xx} + Q(x), &  & 0 < x < \pi, &  & t > 0, \\
                w(0, t)     & = 0,             &  & t > 0,                   \\
                w_x(\pi, t) & = 0,             &  & t > 0
            \end{aligned}
        \end{cases}
    \]
    so that we can find the limiting function \(w_\infty(x) = \lim_{t \to \infty} w(x, t)\) by solving the ODE
    \[
        \begin{cases}
            \begin{aligned}
                -w_\infty''(x) & = Q(x), & 0 < x < \pi, \\
                w_\infty(0)    & = 0,                   \\
                w_\infty'(\pi) & = 0.
            \end{aligned}
        \end{cases}
    \]
    which leads to the limiting function \(u_\infty(x) = w_\infty(x) + v(x)\).

    Let \(v(x) = x\). Then \(v(0) = 0, \> v'(x) = 1, \> v''(x) = 0\), and so
    \[
        \begin{cases}
            \begin{aligned}
                w_t         & = w_{xx} + 3x^2, &  & 0 < x < \pi, &  & t > 0, \\
                w(0, t)     & = 0,             &  & t > 0,                   \\
                w_x(\pi, t) & = 0,             &  & t > 0,                   \\
            \end{aligned}
        \end{cases}
    \]
    and
    \[
        \begin{cases}
            \begin{aligned}
                -w_\infty''(x) & = 3x^2, & 0 < x < \pi, \\
                w_\infty(0)    & = 0,                   \\
                w_\infty'(\pi) & = 0.
            \end{aligned}
        \end{cases}
    \]

    \[w_\infty''(x) = -3x^2 \implies w_\infty'(x) = -x^3 + c_1 \implies w_\infty(x) = -\frac{x^4}{4} + c_1x + c_2.\]
    \[w_\infty(0) = 0 \implies c_2 = 0, \quad w_\infty'(\pi) = 0 \implies -\pi^3 + c_1 = 0 \implies c_1 = \pi^3.\]

    So we have
    \[w_\infty(x) = \pi^3x - \frac{x^4}{4}.\]
    and
    \begin{align*}
        u_\infty(x) & = w_\infty(x) + v(x)                       \\
                    & = \pi^3x - \frac{x^4}{4} + x               \\
                    & = \left(\pi^3 + 1\right)x - \frac{x^4}{4}.
    \end{align*}
\end{solution}

\pagebreak

\begin{problem}
Solve the following wave equation with inhomogeneous mixed boundary conditions
\[
    \begin{cases}
        \begin{aligned}
            u_{tt}      & = u_{xx},                        &  & 0 < x < \pi, &  & t > 0, \\
            u(0, t)     & = 0,                             &  & t > 0,                   \\
            u_x(\pi, t) & = \sin\left(\frac{3t}{2}\right), &  & t > 0,                   \\
            u(x, 0)     & = 0,                             &  & 0 < x < \pi,             \\
            u_t(x, 0)   & = 0,                             &  & 0 < x < \pi.
        \end{aligned}
    \end{cases}
\]
\end{problem}
\begin{solution}
    First, we devise the functions \(w(x, t)\) and \(v(x)\) such that
    \[w(x, t) = u(x, t) - v(x, t)\]
    and
    \[
        \begin{cases}
            \begin{aligned}
                w_{tt} - w_{xx} & = Q(x, t), &  & 0 < x < \pi, &  & t > 0, \\
                w(0, t)         & = 0,       &  & t > 0,                   \\
                w_x(\pi, t)     & = 0,       &  & t > 0.
            \end{aligned}
        \end{cases}
    \]
    We choose \(v(x, t) = \sin\left(\frac{3t}{2}\right)x\). Then
    \[
        \begin{cases}
            \begin{aligned}
                w_{tt} - w_{xx} & = \frac{9}{4}\sin\left(\frac{3t}{2}\right)x, &  & 0 < x < \pi, &  & t > 0, \\
                w(0, t)         & = 0,                                         &  & t > 0,                   \\
                w_x(\pi, t)     & = 0,                                         &  & t > 0,                   \\
                w(x, 0)         & = 0,                                         &  & 0 < x < \pi,             \\
                w_t(x, 0)       & = -\frac{3x}{2},                             &  & 0 < x < \pi.
            \end{aligned}
        \end{cases}
    \]

    We can now solve this using the eigenfunction method.

    First, we show that with these homogeneous mixed boundary conditions in \(x\), the solutions are of the form
    \[w(x, t) \stackrel{L^2}{=} \sum_{n = 1}^\infty A_n(t) \sin(nx).\]
    Using separation of variables and letting \(w(x, t) = X(x)T(t)\), we arrive at the ODE in \(x\)
    \[
        \begin{cases}
            \begin{aligned}
                X''(x)  & = \lambda X(x), &  & 0 < x < \pi, \\
                X(0)    & = 0,            &  & x = 0,       \\
                X'(\pi) & = 0,            &  & x = \pi.
            \end{aligned}
        \end{cases}
    \]
    We can let \(\omega > 0\) and consider the three cases \(\lambda = 0\), \(\lambda = \omega^2 > 0\), and \(\lambda = -\omega^2 < 0\).

    \textbf{Case 1: \(\lambda = 0\).}
    \[X''(x) = 0 \implies X(x) = c_1 + c_2x\]
    \[X(0) = 0 \implies c_1 = 0, \quad X'(\pi) = 0 \implies c_2 = 0.\]
    So, there are only trivial solutions in this case.

    \textbf{Case 2: \(\lambda = \omega^2 > 0\).}
    \[X''(x) = \omega^2 X(x) \implies X(x) = c_1e^{\omega x} + c_2e^{-\omega x}\]
    \[X'(x) = c_1\omega e^{\omega x} - c_2\omega e^{-\omega x}\]
    \[X(0) = 0 \implies c_1 + c_2 = 0 \implies c_2 = -c_1\]
    \[X'(\pi) = 0 \implies c_1\omega e^{\omega \pi} + c_1\omega e^{-\omega \pi} = 0 \implies c_1 = 0\]
    So, there are only trivial solutions in this case.

    \textbf{Case 3: \(\lambda = -\omega^2 < 0\).}
    \[X''(x) = -\omega^2 X(x) \implies X(x) = c_1\cos(\omega x) + c_2\sin(\omega x)\]
    \[X'(x) = -c_1\omega\sin(\omega x) + c_2\omega\cos(\omega x)\]
    \[X(0) = 0 \implies c_1 = 0\]
    \[X'(\pi) = 0 \implies c_2\omega\cos(\omega \pi) = 0 \implies \omega = n + \frac{1}{2}, \> n = 0, 1, 2, \ldots\]
    This admits the solutions
    \[X_n(x) = c_n\sin\left(\Bigl(n + \frac{1}{2}\Bigr)x\right).\]

    With the ODE in \(t\), we now have
    \[T''(t) = -\left(n + \frac{1}{2}\right)^2 T(t) \implies T(t) = a_n\cos\left(\Bigl(n + \frac{1}{2}\Bigr)t\right) + b_n\sin\left(\Bigl(n + \frac{1}{2}\Bigr)t\right).\]
    and the general solution is
    \[w(x, t) = \sum_{n = 0}^\infty \left(a_n\cos\Bigl(\bigl(n + \frac{1}{2}\bigr)t\Bigr) + b_n\sin\Bigl(\bigl(n + \frac{1}{2}\bigr)t\Bigr)\right) \sin\Bigl(\bigl(n + \frac{1}{2}\bigr)x\Bigr).\]

    Now that we have shown that \(w(x, t)\) can be represented as a Fourier sine series in \(x\), we can find the coefficients \(A_n(t)\) using the formula derived in lecture for the wave equation with a source and inhomogeneous mixed initial conditions. Recall that for \(\tilde{u}(x, t) = \sum_{n = 1}^\infty A_n(t) \sin(nx)\), we have
    \[
        \begin{cases}
            \begin{aligned}
                \tilde{u}_{tt} - \tilde{u}_{xx} & = Q(x, t), &  & 0 < x < \pi, &  & t > 0, \\
                \tilde{u}(x, 0)                 & = f(x),    &  & 0 < x < \pi,             \\
                \tilde{u}_t(x, 0)               & = g(x),    &  & 0 < x < \pi.
            \end{aligned}
        \end{cases}
    \]
    where we decompose \(Q(x, t)\), \(f(x)\), and \(g(x)\) into their Fourier sine series as
    \[Q(x, t) = \sum_{n = 1}^\infty q_n(t) \sin(nx), \quad f(x) = \sum_{n = 1}^\infty a_n \sin(nx), \quad g(x) = \sum_{n = 1}^\infty b_n \sin(nx)\]
    which gives us the ODE
    \[
        \begin{cases}
            \begin{aligned}
                A_n''(t) + n^2A_n(t) & = q_n(t), &  & t > 0, \\
                A_n(0)               & = a_n,    &  &        \\
                A_n'(0)              & = b_n.    &  &
            \end{aligned}
        \end{cases}
    \]
    We recall the formula for \(A_n(t)\) is
    \[A_n(t) = \left(a_n - \frac{1}{n}\int_0^t q_n(s)\sin(ns) \dd{s}\right)\cos(nt) + \left(\frac{b_n}{n} + \frac{1}{n}\int_0^t q_n(s)\cos(ns) \dd{s}\right)\sin(nt).\]

    Here, we have \(Q(x, t) = \frac{9}{4}\sin\left(\frac{3t}{2}\right)x\), \(f(x) = 0\), and \(g(x) = -\frac{3x}{2}\). Trivially, \(a_n = 0\). We can find the other Fourier coefficients:
    \begin{align*}
        q_n(t) & = \frac{2}{\pi} \int_0^\pi \frac{9}{4}\sin\left(\frac{3t}{2}\right)x \sin(nx) \dd{x}                          \\
               & = \frac{9}{2\pi} \sin\left(\frac{3t}{2}\right) \int_0^\pi x\sin(nx) \dd{x}                                    \\
               & = \frac{9}{2\pi} \sin\left(\frac{3t}{2}\right) \left[-\frac{x\cos(nx)}{n} + \frac{\sin(nx)}{n^2}\right]_0^\pi \\
               & = \frac{9}{2\pi} \sin\left(\frac{3t}{2}\right) \left(-\frac{\pi}{n}\left(-1\right)^n\right)                   \\
               & = -(-1)^n\frac{9}{2n}\sin\left(\frac{3t}{2}\right).
    \end{align*}
    \begin{align*}
        b_n & = \frac{2}{\pi} \int_0^\pi -\frac{3x}{2} \sin(nx) \dd{x}                        \\
            & = -\frac{3}{\pi} \int_0^\pi x\sin(nx) \dd{x}                                    \\
            & = -\frac{3}{\pi} \left[-\frac{x\cos(nx)}{n} + \frac{\sin(nx)}{n^2}\right]_0^\pi \\
            & = -\frac{3}{\pi} \left(-\frac{\pi}{n}\left(-1\right)^n\right)                   \\
            & = \frac{3(-1)^n}{n}.
    \end{align*}

    Now, let us calculate the coefficient of the \(\cos(nt)\) term in \(A_n(t)\).
    \begin{align*}
        -\frac{1}{n}\int_0^t q_n(s)\sin(ns) \dd{s} & = -\frac{1}{n}\int_0^t -(-1)^n\frac{9}{2n}\sin\left(\frac{3s}{2}\right)\sin(ns) \dd{s}                                                         \\
                                                   & = (-1)^n\frac{9}{2n^2}\int_0^t \sin\left(\frac{3s}{2}\right)\sin(ns) \dd{s}                                                                    \\
                                                   & = (-1)^n\frac{9}{2n^2}\int_0^t \frac{1}{2}\left[\cos\left(\frac{3s}{2} - ns\right) - \cos\left(\frac{3s}{2} + ns\right)\right] \dd{s}          \\
                                                   & = (-1)^n\frac{9}{4n^2}\int_0^t \cos\left(\frac{3s}{2} - ns\right) - \cos\left(\frac{3s}{2} + ns\right) \dd{s}                                  \\
                                                   & = (-1)^n\frac{9}{4n^2}\left[\frac{2}{3 - 2n}\sin\left(\frac{3s}{2} - ns\right) - \frac{2}{3 + 2n}\sin\left(\frac{3s}{2} + ns\right)\right]_0^t \\
                                                   & = (-1)^n\frac{9}{2n^2}\left(\frac{1}{3 - 2n}\sin\left(\frac{3t}{2} - nt\right) - \frac{1}{3 + 2n}\sin\left(\frac{3t}{2} + nt\right)\right).
    \end{align*}

    And, the coefficient of the \(\sin(nt)\) term in \(A_n(t)\).
    \begin{align*}
        \frac{b_n}{n} + \frac{1}{n}\int_0^t q_n(s)\cos(ns) \dd{s} & = \frac{3(-1)^n}{n^2} + \frac{1}{n}\int_0^t -(-1)^n\frac{9}{2n}\sin\left(\frac{3s}{2}\right)\cos(ns) \dd{s}                                                                                            \\
                                                                  & = \frac{3(-1)^n}{n^2} - (-1)^n\frac{9}{2n^2}\int_0^t \sin\left(\frac{3s}{2}\right)\cos(ns) \dd{s}                                                                                                      \\
                                                                  & = \frac{3(-1)^2}{n^2}\left(1-\frac{3}{2} \int_0^t \sin\left(\frac{3s}{2}\right)\cos(ns) \dd{s}\right)                                                                                                  \\
                                                                  & = \frac{3(-1)^2}{n^2}\left(1-\frac{3}{4}\int_0^t \sin\left(\frac{3s}{2} - ns\right) + \sin\left(\frac{3s}{2} + ns\right) \dd{s}\right)                                                                 \\
                                                                  & = \frac{3(-1)^2}{n^2}\left(1+\frac{3}{2}\left[\frac{1}{3 - 2n}\cos\left(\frac{3s}{2} - ns\right) + \frac{1}{3 + 2n}\cos\left(\frac{3s}{2} + ns\right)\right]_0^t\right)                                \\
                                                                  & = \frac{3(-1)^2}{n^2}\left(1+\frac{3}{2}\left[\frac{1}{3 - 2n}\left(\cos\left(\frac{3t}{2} - nt\right) - 1\right) + \frac{1}{3 + 2n}\left(\cos\left(\frac{3t}{2} + nt\right) - 1\right)\right]\right).
    \end{align*}

    So, we have
    \[w(x, t) = \sum_{n = 1}^\infty A_n(t) \sin(nx)\]
    with \(A_n\) given with the coefficients calculated above.

    Finally, we have
    \[u(x, t) = w(x, t) + v(x, t) = \sin\left(\frac{3t}{2}\right)x + \sum_{n = 1}^\infty A_n(t) \sin(nx).\]


\end{solution}

\end{document}
