\documentclass[plain]{pset}
\usepackage{dylan}

\title{Homework 5}
\author{Dylan Hu}
\prof{Professor Zhuolun Yang}
\course{APMA 0360 --- Partial Differential Equations}
\date{March 17, 2024}

\begin{document}

\begin{multicols}{2}
    \raggedcolumns{}
    \maketitle
    \columnbreak{}
    \tableofcontents
\end{multicols}

\setlength{\parskip}{1em}
\setlength{\parindent}{0pt}

\pagebreak

\begin{problem}
\[\]
\vspace{-4em}
\begin{enumerate}[a.]
    \item Verify \(\cos(nx)\)and \(\cos(mx)\) are orthogonal on \((0, \pi)\) when \(n \neq m\) and \(n, m \in \mathbb{N}\). That is
          \[
              \int_0^\pi \cos(nx)\cos(mx) \dd x= 0.
          \]
    \item Verify \(\sin(nx)\) and \(\sin(mx)\) are orthogonal on \((-\pi, \pi)\) when \(n, m \in \mathbb{N}\). That is
          \[
              \int_{-\pi}^\pi \sin(nx)\cos(mx) \dd x= 0.
          \]
\end{enumerate}
\end{problem}
\begin{solution}
    \[\]
    \vspace{-4em}
    \begin{enumerate}[a.]
        \item
              \begin{align*}
                  \int_0^\pi \cos(nx)\cos(mx) \dd x & = \frac{1}{2}\int_0^\pi \left(\cos((n+m)x) + \cos((n-m)x)\right) \dd x                 \\
                                                    & = \frac{1}{2}\left[\frac{1}{n+m}\sin((n+m)x) + \frac{1}{n-m}\sin((n-m)x) \right]_0^\pi \\
                                                    & = \frac{1}{2}\left[\frac{1}{n+m}\sin((n+m)\pi) + \frac{1}{n-m}\sin((n-m)\pi) \right]   \\
                                                    & = 0
              \end{align*}
              We have used the fact that \(\sin(m\pi) = \sin(n\pi)= 0\) for \(m, n \in \mathbb{N}\) and that \(n \neq m \implies n \pm m \neq 0\).
        \item
              \begin{align*}
                  \int_{-\pi}^\pi \sin(nx)\cos(mx) \dd x & = \frac{1}{2} \int_{-\pi}^\pi \left(\sin((n+m)x) + \sin((n-m)x)\right) \dd x                                                 \\
                                                         & = \frac{1}{2} \int_{-\pi}^\pi \left(\sin((n+m)x) \right) \dd x + \frac{1}{2} \int_{-\pi}^\pi \left(\sin((n-m)x)\right) \dd x \\
                                                         & = 0 + 0 = 0
              \end{align*}
              We use the fact that \(\sin\) is an odd function and \((-\pi, \pi)\) is symmetric, so \(\int_{-\pi}^\pi \sin((n+m)x) \dd x = 0\) and \(\int_{-\pi}^\pi \sin((n-m)x) \dd x = 0\).
    \end{enumerate}
\end{solution}

\pagebreak

\begin{problem}
Solve
\[
    \begin{cases}
        \begin{aligned}
            u_t - u_{xx}            & = 0,   &  & 0 < x < \pi, & t > 0, \\
            u_x(0, t) = u_x(\pi, t) & = 0,   &  & t > 0,                \\
            u(x, 0)                 & = x^2, &  & 0 < x < \pi.
        \end{aligned}
    \end{cases}
\]
explicitly.
\end{problem}

\begin{solution}
    \[u(x, t) = X(x)T(t)\]
    We recall from lecture that the solution to \(X''(x) = \lambda X(x)\) with the von Neumann boundary conditions is
    \[
        X(x) = c \cos(nx), \quad n \geq 0
    \]
    where \(c\) is a constant and \(\lambda = -n^2\). Then we also have
    \[T'(t) = \lambda T(t) \implies T(t) = e^{\lambda t} = e^{-n^2t}\]
    So the general solution is
    \[u(x, t) = \sum_{n=0}^\infty a_n \cos(nx) e^{-n^2t}\]
    We can solve for the \(c_n\) using the initial condition:
    \[u(x, 0) = \sum_{n=0}^\infty a_n \cos(nx) = x^2\]
    We recognize the sum as a Fourier series and solve for \(a_n\):
    \[a_n = \frac{2}{\pi} \int_0^\pi x^2 \cos(nx) \dd x = \frac{2}{\pi} \left[\frac{2x^2}{n^2} \sin(nx) + \frac{2}{n}x\cos(nx)\right]_0^\pi = \frac{4}{n^2}(-1)^n\]
    We notice that for \(n = 0\) our solution is undefined, so we must add the \(n = 0\) term separately:
    \[a_0 = \frac{1}{\pi} \int_0^\pi x^2 \dd x = \frac{1}{\pi} \left[\frac{x^3}{3}\right]_0^\pi = \frac{\pi^2}{3}\]
    So the solution is
    \[u(x, t) = \frac{\pi^2}{3} + \sum_{n=1}^\infty \frac{4}{n^2}(-1)^n \cos(nx) e^{-n^2t}\]
\end{solution}

\pagebreak

\begin{problem}
Write the Fourier sine series (as \(\sum a_n \sin(nx)\)) of the function
\[f(x) = \begin{cases}
        \begin{aligned}
            -1, &  & 0 < x < \frac{\pi}{2},   \\
            1,  &  & \frac{\pi}{2} < x < \pi.
        \end{aligned}
    \end{cases}\]
For each \(n \in \mathbb{N}\), what is the value \(a_n \sin(n\pi/2)\)?

You can separate the cases when \(n = 4k, 4k+1, 4k+2, 4k+3\) for \(k \in \mathbb{N} \cup \{0\}\).
\end{problem}
\begin{solution}
    \[f(x) = \sum_{n=1}^\infty a_n \sin(nx)\]
    \begin{align*}
        a_n & = \frac{2}{\pi} \int_0^\pi f(x) \sin(nx) \dd x                                                                                                  \\
            & = \frac{2}{\pi} \left[\int_0^{\pi/2} (-1)\sin(nx) \dd x + \int_{\pi/2}^\pi \sin(nx) \dd x\right]                                              \\
            & = \frac{2}{\pi} \left[\frac{1}{n} \cos(nx)\Big|_0^{\pi/2} - \frac{1}{n} \cos(nx)\Big|_{\pi/2}^\pi\right]                                      \\
            & = \frac{2}{\pi} \left[\frac{\cos(n\pi/2)}{n} - \frac{\cos(0)}{n} - \frac{\cos(n\pi)}{n} + \frac{\cos(n\pi/2)}{n}\right]                    \\
            & = \frac{2}{\pi} \left[\frac{2\cos(n\pi/2) - (-1)^n - 1}{n}\right]                                                                           \\
    \end{align*}
    We now have
    \[f(x) = \sum_{n=1}^\infty \frac{2}{\pi} \left[\frac{2\cos(n\pi/2) - (-1)^n - 1}{n}\right] \sin(nx)\]
    We can separate the cases when \(n = 4k, 4k+1, 4k+2, 4k+3\) for \(k \in \mathbb{N} \cup \{0\}\).

    \textbf{Case 1:} \(n = 4k\) (note that we can ignore \(k = 0\) in this case since \(n = 0\) is not in the sum)
    \[a_n \sin(n\pi/2) = \frac{2}{\pi} \left[\frac{2\cos(2k\pi) - (-1)^{4k} - 1}{4k}\right]\sin(2k\pi) = \frac{2}{\pi} \left[\frac{2 - 1 - 1}{4k}\right]\sin(2k\pi) = 0\]

    \textbf{Case 2:} \(n = 4k+1\)
    \[a_n \sin(n\pi/2) = \frac{2}{\pi} \left[\frac{2\cos(2k\pi + \pi/2) - (-1)^{4k+1} - 1}{4k+1}\right]\sin(2k\pi + \pi/2) = \frac{2}{\pi} \left[\frac{2(0) + 1 - 1}{4k+1}\right]\sin(2k\pi + \pi/2) = 0\]

    \textbf{Case 3:} \(n = 4k+2\)
    \[a_n \sin(n\pi/2) = \frac{2}{\pi} \left[\frac{2\cos(2k\pi + \pi) - (-1)^{4k+2} - 1}{4k+2}\right]\sin(2k\pi + \pi) = \frac{2}{\pi} \left[\frac{2(-1) - 1 - 1}{4k+2}\right](0) = 0\]

    \textbf{Case 4:} \(n = 4k+3\)
    \[a_n \sin(n\pi/2) = \frac{2}{\pi} \left[\frac{2\cos(2k\pi + 3\pi/2) - (-1)^{4k+3} - 1}{4k+3}\right]\sin(2k\pi + 3\pi/2) = \frac{2}{\pi} \left[\frac{2(0) - (-1) - 1}{4k+3}\right]\sin(2k\pi + 3\pi/2) = 0\]

    Therefore, we have \(a_n \sin(n\pi/2) = 0\) for all \(n \in \mathbb{N}\).



\end{solution}

\pagebreak

\begin{problem}
    Solve
    \[
        \begin{cases}
            \begin{aligned}
                u_{tt} - u_{xx} & = 0, &  & 0 < x < \pi, & t > 0, \\
                u(0, t) = u(\pi, t) & = 0, &  & t > 0, \\
                u(x, 0) & = x^2, &  & 0 < x < \pi, \\
                u_t(x, 0) & = sin(2x), &  & 0 < x < \pi.
            \end{aligned}
        \end{cases}
    \]
    explicitly.
\end{problem}
\begin{solution}
    \[u(x, t) = X(x)T(t)\]
    \[\frac{T''(t)}{T(t)} = \frac{X''(x)}{X(x)} = \lambda\]
    Considering the ODE in \(x\):
    \textbf{Case 1:} \(\lambda = 0\)
    \[X''(x) = 0 \implies X(x) = a_1 + a_2x\]
    Applying the boundary conditions:
    \[X(0) = a_1 = 0 \implies a_1 = 0\]
    \[X(\pi) = a_1 + a_2\pi = 0 \implies a_2 = 0\]

    \textbf{Case 2:} \(\lambda = -\omega^2, \omega > 0\)
    \[X''(x) = -\omega^2X(x) \implies X(x) = a_1\cos(\omega x) + a_2\sin(\omega x)\]
    Applying the boundary conditions:
    \[X(0) = a_1 = 0 \implies a_1 = 0\]
    \[X(\pi) = a_2\sin(\omega \pi) = 0 \implies \omega = n, \quad n \in \mathbb{N}\]

    So the general solution for \(X(x)\) is
    \[X(x) = \sum_{n=1}^\infty a_n\sin(nx)\]

    \[T''(t) = -n^2T(t) \implies T(t) = c_1\cos(nt) + c_2\sin(nt)\]
    So the general solution is
    \[u(x, t) = \sum_{n=1}^\infty \left(a_n\sin(nx) + b_n\sinh(nx)\right)\left(c_1\cos(nt) + c_2\sin(nt)\right)\]

\end{solution}



\end{document}
