\documentclass[plain]{pset}
\usepackage{dylan}

\title{Homework 4}
\author{Dylan Hu}
\prof{Professor Zhuolun Yang}
\course{APMA 0360 --- Partial Differential Equations}
\date{March 10, 2024}

\begin{document}

\begin{multicols}{2}
    \raggedcolumns{}
    \maketitle
    \columnbreak{}
    \tableofcontents
\end{multicols}

\setlength{\parskip}{1em}

\pagebreak

\begin{problem}
Verify the superposition principle: if \(u_1\) and \(u_2\) are two solutions to
\[
    \begin{cases}
        \begin{aligned}
            u_t - u_{xx}        & = 0, & 0 & < x < \pi, & t > 0, \\
            u(0, t) = u(\pi, t) & = 0, & t & > 0                 \\
        \end{aligned}
    \end{cases}
\]
then so is \(a_1u_1 + a_2u_2\) for any constants \(a_1, a_2\). Is it still true if the boundary condition is \(u(0, t) = u(\pi, t) = 1\)?
\end{problem}
\begin{solution}

\end{solution}

\pagebreak

\begin{problem}
Use separation of variables to solve
\[
    \begin{cases}
        \begin{aligned}
            u_t - u_{xx}            & = 0,             & 0 & < x < \pi, & t > 0, \\
            u_x(0, t) = u_x(\pi, t) & = 0,             & t & > 0,                \\
            u(x, 0)                 & = 1 + 2\cos(3x), & 0 & < x < \pi           \\
        \end{aligned}
    \end{cases}
\]
explicitly. You don't need to go through all 3 cases.
\end{problem}
\begin{solution}

\end{solution}

\pagebreak

\begin{problem}
Consider the following heat-like equation:
\[
    \begin{cases}
        \begin{aligned}
            tu_t - u_{xx}             & = 0, & 0 & < x < \pi, & t > 0, \\
            u(0, t)       = u(\pi, t) & = 0, & t & > 0,                \\
        \end{aligned}
    \end{cases}
\]
Use separation of variables to write the solution as an infinite series. You don't need to go through all 3 cases.
\end{problem}
\begin{solution}

\end{solution}

\pagebreak

\begin{problem}
Consider the heat equation with mixed boundary conditions
\[
    \begin{cases}
        \begin{aligned}
            u_t - u_{xx} & = 0, & 0 & < x < \pi, & t > 0, \\
            u_x(0, t)    & = 0, & t & > 0,                \\
            u(\pi, t)    & = 0, & t & > 0.                \\
        \end{aligned}
    \end{cases}
\]
Use separation of variables and go through all 3 cases to write the solution as an infinite series.
\end{problem}
\begin{solution}

\end{solution}

\end{document}
