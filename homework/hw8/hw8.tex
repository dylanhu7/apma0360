\documentclass[plain]{pset}
\usepackage{dylanhu}

\title{Homework 8}
\author{Dylan Hu}
\prof{Professor Zhuolun Yang}
\course{APMA 0360 --- Partial Differential Equations}
\date{April 28, 2024}

\begin{document}

\maketitle

\pagebreak

\begin{problem}
\leavevmode
\begin{enumerate}[(a)]
    \item Construct a twice differentiable function \(v(x,t)\) such that
          \[v(0,t) = g(t), \quad v_x(\pi, t) = h(t).\]
    \item Construct a twice differentiable function \(v(x,t)\) such that
          \[v_x(0,t) = g(t), \quad v_x(\pi, t) = h(t).\]
\end{enumerate}
For both parts, show some computations to justify that the functions you construct do
satisfy those boundary conditions.
\end{problem}
\begin{solution}

\end{solution}

\pagebreak

\begin{problem}
Solve the heat equation with source:
\[
    \begin{cases}
        \begin{aligned}
            u_t - u_{xx}        & = e^{-t}\sin(3x), &  & 0 < x < \pi, & t > 0, \\
            u(0, t) = u(\pi, t) & = 0,              &  & t > 0,                \\
            u(x, 0)             & = 1,              &  & 0 < x < \pi.
        \end{aligned}
    \end{cases}
\]
\end{problem}
\begin{solution}

\end{solution}

\pagebreak

\begin{problem}
Solve the heat equation with inhomogeneous Dirichlet boundary condition:
\[
    \begin{cases}
        \begin{aligned}
            u_t - u_{xx} & = 0, &  & 0 < x < \pi, & t > 0, \\
            u(0, t)      & = 0, &  & t > 0,                \\
            u(\pi, t)    & = t, &  & t > 0,                \\
            u(x, 0)      & = 0, &  & 0 < x < \pi.
        \end{aligned}
    \end{cases}
\]
\end{problem}
\begin{solution}

\end{solution}

\pagebreak

\begin{problem}
Solve the wave equation with a constant gravitational force:
\[
    \begin{cases}
        \begin{aligned}
            u_{tt} - u_{xx}              & = -1, &  & 0 < x < \pi, & t > 0, \\
            u(0, t)          = u(\pi, t) & = 0,  &  & t > 0,                \\
            u(x, 0)          = u_t(x, 0) & = 0,  &  & 0 < x < \pi.
        \end{aligned}
    \end{cases}
\]
\end{problem}
\begin{solution}

\end{solution}

\end{document}
