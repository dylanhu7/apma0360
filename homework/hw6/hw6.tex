\documentclass[plain]{pset}
\usepackage{dylanhu}

\title{Homework 6}
\author{Dylan Hu}
\prof{Professor Zhuolun Yang}
\course{APMA 0360 --- Partial Differential Equations}
\date{April 7, 2024}

\begin{document}

\maketitle

\pagebreak

\begin{problem}
Assume that \(H(x, t),\ I(x,t)\) satisfy
\begin{align*}
    H_t & = -bHI,                     \\
    I_t & = bHI - \gamma I + DI_{xx},
\end{align*}
Define
\[\tau \coloneqq \gamma t, \quad h \coloneqq \frac{H}{N}, \quad i \coloneqq \frac{I}{N}, \quad R_0 \coloneqq \frac{bN}{\gamma}, \quad d \coloneqq \frac{D}{\gamma},\]
where \(b, \gamma, N, D\) are some positive constants. Show that \(h(x,\tau),\ i(x, \tau)\) satisfy the system
\begin{align*}
    h_\tau & = -R_0hi,              \\
    i_\tau & = R_0hi - i + di_{xx}.
\end{align*}
\textit{Hint: chain rule.}
\end{problem}
\begin{solution}
    \begin{align*}
        h_\tau & = \pdv{h}{\tau}                                                                                        \\
               & = \pdv{h}{t}\pdv{t}{\tau}                                                                              \\
               & = \left(\pdv{}{t}\left(\frac{H}{N}\right)\right)\frac{1}{\gamma}                                       \\
               & = \frac{1}{N}\pdv{H}{t}\frac{1}{\gamma}                                                                \\
               & = \frac{1}{N}\left(-bHI\right)\frac{1}{\gamma}                                                         \\
               & = -\frac{bHI}{N}\frac{R_0}{bN}                                                                         \\
               & = -R_0 \frac{H}{N}\frac{I}{N}                                                                          \\
               & = -R_0 hi                                                                                              \\
               &                                                                                                        \\
        i_\tau & = \pdv{i}{\tau}                                                                                        \\
               & = \pdv{i}{t}\pdv{t}{\tau}                                                                              \\
               & = \left(\pdv{}{t}\left(\frac{I}{N}\right)\right)\frac{1}{\gamma}                                       \\
               & = \frac{1}{N}\pdv{I}{t}\frac{1}{\gamma}                                                                \\
               & = \frac{1}{N}\left(bHI - \gamma I + DI_{xx}\right)\frac{1}{\gamma}                                     \\
               & = \frac{bHI}{N}\frac{R_0}{bN} - \frac{\gamma I}{N}\frac{1}{\gamma} + \frac{DI_{xx}}{N}\frac{1}{\gamma} \\
               & = R_0 hi - \frac{I}{N} + \frac{DI_{xx}}{N\gamma}                                                       \\
               & = R_0 hi - i + d\frac{I_{xx}}{N}                                                                       \\
               & = R_0 hi - i + di_{xx} \quad \qquad \left(i_{xx} = \frac{I_{xx}}{N}\right)
    \end{align*}
\end{solution}

\pagebreak

\begin{problem}
For \((x, y) \neq (0, 0)\), compute the Laplacian of
\[u(x, y) = \ln\left(x^2 + y^2\right)\]
and conclude whether it satisfies the Laplace equation for \((x, y) \neq (0, 0)\).
\end{problem}
\begin{solution}
    \begin{align*}
        u_x      & = \frac{2x}{x^2 + y^2},                                                                        \\
        u_y      & = \frac{2y}{x^2 + y^2},                                                                        \\
        u_{xx}   & = \frac{2(x^2 + y^2) - 2x(2x)}{(x^2 + y^2)^2} = \frac{2y^2 - 2x^2}{(x^2 + y^2)^2},             \\
        u_{yy}   & = \frac{2(x^2 + y^2) - 2y(2y)}{(x^2 + y^2)^2} = \frac{2x^2 - 2y^2}{(x^2 + y^2)^2},             \\
        \Delta u & = u_{xx} + u_{yy} = \frac{2y^2 - 2x^2}{(x^2 + y^2)^2} + \frac{2x^2 - 2y^2}{(x^2 + y^2)^2} = 0.
    \end{align*}
    Thus, \(u(x, y) = \ln\left(x^2 + y^2\right)\) satisfies the Laplace equation for \((x, y) \neq (0, 0)\).
\end{solution}

\pagebreak

\begin{problem}
Use separation of variables to solve
\[
    \begin{cases}
        \begin{aligned}
            u_{xx} + u_{yy} & = 0,             &  & 0 < x, y < \pi, \\
            u_y(x, 0)       & = 0,             &  & 0 < x < \pi,    \\
            u_y(x, \pi)     & = 0,             &  & 0 < x < \pi,    \\
            u(0, y)         & = 0,             &  & 0 < y < \pi,    \\
            u(\pi, y)       & = 1 + 3\cos(2y), &  & 0 < y < \pi.
        \end{aligned}
    \end{cases}
\]
\end{problem}
\begin{solution}
    Let \(u(x, y) = X(x)Y(y)\). Then the Laplace equation gives
    \[-\frac{X''(x)}{X(x)} = \frac{Y''(y)}{Y(y)} = \lambda.\]
    We consider the system
    \[
        \begin{cases}
            \begin{aligned}
                Y''(y)          & = \lambda Y(y), \quad 0 < y < \pi, \\
                Y'(0) = Y'(\pi) & = 0.
            \end{aligned}
        \end{cases}
    \]
    Recall that the general solution for the homogeneous Neumann boundary conditions with \(\lambda = -n^2\) is
    \[Y(y) = c\cos(ny), \quad n = 0, 1, 2, \ldots\]
    for some constant \(c\).

    Note that in the case of \(n = 0\), we have \(Y(y) = c\), which admits a solution
    \[X(x) = a_0 + b_0 x.\]

    We also have
    \[X''(x) = -\lambda X(x) = n^2 X(x),\]
    and we recall the general solution
    \[X(x) = Ae^{nx} + Be^{-nx}\]
    where \(A, B\) are arbitrary constants. Using the hyperbolic sine and cosine functions, we can rewrite this as
    \[X(x) = \tilde{A}\cosh(nx) + \tilde{B}\sinh(nx).\]

    We can now write the general solution with \(u_y(x, 0) = u_y(x, \pi) = 0\) as
    \[u(x, y) = a_0 + b_0 x + \sum_{n=1}^\infty \bigl(a_n \cosh(nx) + b_n \sinh(nx)\bigr)\cos(ny)\]

    The boundary condition \(u(0, y) = 0\) gives \(x = 0, \ \cosh(0) = 1, \ \sinh(0) = 0\), so
    \begin{align*}
        u(0, y) & = a_0 + \sum_{n=1}^\infty a_n \cos(ny) \\
                & = \sum_{n=0}^\infty a_n \cos(ny)       \\
                & = 0.
    \end{align*}
    We recognize this as a Fourier series, so
    \[a_n = \frac{2}{\pi} \int_0^\pi 0 \cdot \cos(ny) \dd y = 0.\]

    With the boundary condition \(u(\pi, y) = 1 + 3\cos(2y)\), we have
    \[u(\pi, y) = \pi b_0 + \sum_{n=1}^\infty b_n \sinh(n\pi)\cos(ny) = 1 + 3\cos(2y).\]
    As it is a constant, we can let \(b_n \sinh(n\pi) = \tilde{b}_n\). Then
    \begin{align*}
        \tilde{b}_n & = \frac{2}{\pi} \int_0^\pi \bigl(1 + 3\cos(2y) - \pi b_0\bigr)\cos(ny) \dd y                                                                                                              \\
                    & = \frac{2}{\pi} \int_0^\pi \cos(ny) \dd y + \frac{6}{\pi} \int_0^\pi \cos(2y)\cos(ny) \dd y - \pi b_0 \frac{2}{\pi} \int_0^\pi \cos(ny) \dd y                                             \\
                    & = \frac{2}{\pi} \left[\frac{1}{n}\sin(ny)\right]_0^\pi + \frac{6}{\pi} \int_0^\pi \frac{1}{2}\left(\cos((n+2)y) + \cos((n-2)y)\right) \dd y - 2b_0 \left[\frac{1}{n}\sin(ny)\right]_0^\pi \\
                    & = 0 + \frac{3}{\pi} \left[\frac{1}{n+2}\sin((n+2)y) + \frac{1}{n-2}\sin((n-2)y)\right]_0^\pi - 0                                                                                          \\
                    & = 0.
    \end{align*}

    We see that \(b_n = \frac{\tilde{b}_n}{\sinh(n\pi)} = 0\) for all \(n \neq 2\). Thus, we have
    \[u(x, y) = b_0 x + b_2 \sinh(2x)\cos(2y).\]

    We can again apply the \(u(\pi, y) = 1 + 3\cos(2y)\) boundary condition to solve for \(b_0\):
    \[u(\pi, y) = \pi b_0 + b_2 \sinh(2\pi)\cos(2y) = 1 + 3\cos(2y).\]

    Comparing terms, we see that \(b_0 = \frac{1}{\pi}\) and \(b_2 = \frac{3}{\sinh(2\pi)}\). Thus, the solution is
    \[u(x, y) = \frac{x}{\pi} + \frac{3\sinh(2x)}{\sinh(2\pi)}\cos(2y).\]
\end{solution}

\pagebreak

\begin{problem}
Use separation of variables to solve
\[
    \begin{cases}
        \begin{aligned}
            u_{xx} + u_{yy} & = 0,   &  & 0 < x, y < \pi, \\
            u(x, 0)         & = 0,   &  & 0 < x < \pi,    \\
            u(x, \pi)       & = 100, &  & 0 < x < \pi,    \\
            u(0, y)         & = 0,   &  & 0 < y < \pi,    \\
            u(\pi, y)       & = 100, &  & 0 < y < \pi.
        \end{aligned}
    \end{cases}
\]
\end{problem}
\begin{solution}
    As we do not have either the homogeneous Dirichlet or Neumann boundary conditions, we first use the superposition principle to write
    \[u(x, y) = u_1(x, y) + u_2(x, y)\]
    where
    \[
        \begin{cases}
            \begin{aligned}
                u_{1xx} + u_{1yy} & = 0,   &  & 0 < x, y < \pi, \\
                u_1(x, 0)         & = 0,   &  & 0 < x < \pi,    \\
                u_1(x, \pi)       & = 100, &  & 0 < x < \pi,    \\
                u_1(0, y)         & = 0,   &  & 0 < y < \pi,    \\
                u_1(\pi, y)       & = 0,   &  & 0 < y < \pi.
            \end{aligned}
        \end{cases}
    \]
    and
    \[
        \begin{cases}
            \begin{aligned}
                u_{2xx} + u_{2yy} & = 0,   &  & 0 < x, y < \pi, \\
                u_2(x, 0)         & = 0,   &  & 0 < x < \pi,    \\
                u_2(x, \pi)       & = 0,   &  & 0 < x < \pi,    \\
                u_2(0, y)         & = 0,   &  & 0 < y < \pi,    \\
                u_2(\pi, y)       & = 100, &  & 0 < y < \pi.
            \end{aligned}
        \end{cases}
    \]
    Then we can proceed using separation of variables to solve each of these equations separately.

    \subsubsection*{Part 1: \(u_1(x, y)\)}
    Let \(u_1(x, y) = X(x)Y(y)\). Then the Laplace equation gives
    \[\frac{X''(x)}{X(x)} = -\frac{Y''(y)}{Y(y)} = \lambda.\]
    We consider the system
    \[
        \begin{cases}
            \begin{aligned}
                X''(x)        & = \lambda X(x), \quad 0 < x < \pi, \\
                X(0) = X(\pi) & = 0,
            \end{aligned}
        \end{cases}
    \]
    The general solution for the homogeneous Dirichlet boundary conditions with \(\lambda = -n^2\) is
    \[X(x) = c\sin(nx), \quad n = 1, 2, 3, \ldots\]
    for some constant \(c\).

    We also have
    \[Y''(y) = -\lambda Y(y) = n^2 Y(y),\]
    and we recall the general solution
    \[Y(y) = Ae^{ny} + Be^{-ny}\]
    where \(A, B\) are arbitrary constants. Using the hyperbolic sine and cosine functions, we can rewrite this as
    \[Y(y) = \tilde{A}\cosh(ny) + \tilde{B}\sinh(ny).\]

    We can now write the general solution with \(u_1(0, y) = u_1(\pi, y) = 0\) as
    \[u_1(x, y) = \sum_{n=1}^\infty \bigl(a_n \cosh(ny) + b_n \sinh(ny)\bigr)\sin(nx)\]

    The boundary condition \(u_1(x, 0) = 0\) gives \(y = 0, \ \cosh(0) = 1, \ \sinh(0) = 0\), so
    \[u_1(x, 0) = \sum_{n=1}^\infty a_n \sin(nx) = 0\]

    We recognize this as a Fourier sine series, so
    \[a_n = \frac{2}{\pi} \int_0^\pi 0 \cdot \sin(nx) \dd x = 0.\]

    With the boundary condition \(u_1(x, \pi) = 100\), we have
    \[u_1(x, \pi) = \sum_{n=1}^\infty b_n \sinh(n\pi)\sin(nx) = 100.\]

    As it is a constant, we can let \(b_n \sinh(n\pi) = \tilde{b}_n\). Then
    \begin{align*}
        \tilde{b}_n & = \frac{2}{\pi} \int_0^\pi 100 \cdot \sin(nx) \dd x                        \\
                    & = \frac{200}{\pi} \int_0^\pi \sin(nx) \dd x                                \\
                    & = \frac{200}{\pi} \left[-\frac{1}{n}\cos(nx)\right]_0^\pi                  \\
                    & = \frac{200}{\pi} \left[-\frac{1}{n}\cos(n\pi) + \frac{1}{n}\cos(0)\right] \\
                    & = \frac{200}{\pi} \left[\frac{1 - (-1)^n}{n}\right].
    \end{align*}

    Then
    \[b_n = \frac{\tilde{b}_n}{\sinh(n\pi)} = \frac{200}{\pi} \left[\frac{1 - (-1)^n}{n}\right]\frac{1}{\sinh(n\pi)}\]
    for all \(n\).

    So, our solution is
    \[u_1(x, y) = \frac{200}{\pi}\sum_{n=1}^\infty \left[\frac{1 - (-1)^n}{n}\right] \frac{\sinh(ny)}{\sinh(n\pi)}\sin(nx).\]

    \subsubsection*{Part 2: \(u_2(x, y)\)}
    By symmetry, we can see that the solution to \(u_2(x, y)\) will be the same as \(u_1(x, y)\) but with \(x\) and \(y\) interchanged:
    \[u_2(x, y) = \frac{200}{\pi}\sum_{n=1}^\infty \left[\frac{1 - (-1)^n}{n}\right] \frac{\sinh(nx)}{\sinh(n\pi)}\sin(ny).\]

    \subsubsection*{Combining the solutions}
    By the superposition principle, the solution to the original problem is
    \begin{align*}
        u(x, y) & = u_1(x, y) + u_2(x, y)                                                                                                                                                                                               \\
                & = \frac{200}{\pi}\sum_{n=1}^\infty \left[\frac{1 - (-1)^n}{n}\right] \frac{\sinh(ny)}{\sinh(n\pi)}\sin(nx) + \frac{200}{\pi}\sum_{n=1}^\infty \left[\frac{1 - (-1)^n}{n}\right] \frac{\sinh(nx)}{\sinh(n\pi)}\sin(ny) \\
                & = \frac{200}{\pi}\sum_{n=1}^\infty \left[\frac{1 - (-1)^n}{n}\right] \frac{\sinh(ny)\sin(nx) + \sinh(nx)\sin(ny)}{\sinh(n\pi)}.
    \end{align*}
\end{solution}

\end{document}
