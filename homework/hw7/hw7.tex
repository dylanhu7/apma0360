\documentclass[plain]{pset}
\usepackage{dylanhu}

\title{Homework 7}
\author{Dylan Hu}
\prof{Professor Zhuolun Yang}
\course{APMA 0360 --- Partial Differential Equations}
\date{April 21, 2024}

\begin{document}

\maketitle

\pagebreak

\begin{problem}
Find the solution to the boundary value problem on a wedge:
\[
    \begin{cases}
        \begin{aligned}
            \Delta u     & = 0,         &  & 0 < r < 1, \quad 0 < \theta < \pi/2, \\
            u(r, 0)      & = 0,         &  & 0 < r < 1,                           \\
            u(r, \pi/2)  & = 0,         &  & 0 < r < 1,                           \\
            u(1, \theta) & = f(\theta), &  & 0 < \theta < \pi/2.
        \end{aligned}
    \end{cases}
\]
\begin{hint}
    \emph{Be careful that \(\theta \in (0, \pi/2)\), we do not have periodic boundary condition in \(\theta\). And the formulas for Fourier coefficients are slightly different than what we have when \(\theta \in (0, \pi)\) or \(\theta \in (0, 2\pi)\).}
\end{hint}
\end{problem}
\begin{solution}
    First, we recall that the Laplacian in polar coordinates is given by
    \[
        \Delta u = u_{rr} + \frac{u_r}{r} + \frac{u_{\theta\theta}}{r^2}.
    \]
    Using separation of variables, we assume that \(u(r, \theta) = R(r)\Theta(\theta)\). Then Laplace's equation becomes
    \[
        R''\Theta + \frac{R'\Theta}{r} + \frac{R\Theta''}{r^2} = 0 \iff \frac{r^2R'' + rR'}{R} = -\frac{\Theta''}{\Theta} = \lambda.
    \]
    The boundary conditions in \(\theta\) give \(\Theta(0) = \Theta\left(\frac{\pi}{2}\right) = 0\). We solve the ODE in \(\theta\) first.
    \[
        \begin{cases}
            \begin{aligned}
                \Theta'' + \lambda\Theta         & = 0, &  & 0 < \theta < \frac{\pi}{2}, \\
                \Theta(0)                        & = 0, &  & 0 < \theta < \frac{\pi}{2}, \\
                \Theta\left(\frac{\pi}{2}\right) & = 0, &  & 0 < \theta < \frac{\pi}{2}.
            \end{aligned}
        \end{cases}
    \]

    \textbf{Case 1:} \(\lambda < 0\). Let \(\lambda = -\omega^2\) where \(\omega > 0\).
    \[\Theta''(\theta) = \omega^2 \Theta(\theta) \implies \Theta(\theta) = Ae^{\omega\theta} + Be^{-\omega\theta}.\]
    The boundary conditions give
    \[
        \begin{cases}
            \begin{aligned}
                \Theta(0)                        & = A + B = 0,                                \\
                \Theta\left(\frac{\pi}{2}\right) & = Ae^{\omega\pi/2} + Be^{-\omega\pi/2} = 0.
            \end{aligned}
        \end{cases}
    \]
    By substituting the first equation into the second, we find that \(A\left(e^{\omega\pi/2} - e^{-\omega\pi/2}\right) = 0\), and we see that the only solution is \(A = B = 0 \implies \Theta(\theta) = 0\).

    \textbf{Case 2:} \(\lambda = 0\). Then \(\Theta'' = 0 \implies \Theta(\theta) = A + B\theta\). The boundary conditions give \(A = A + B\frac{\pi}{2} = 0\), and we find that \(A = B = 0 \implies \Theta(\theta) = 0\).

    \textbf{Case 3:} \(\lambda > 0\). Let \(\lambda = \omega^2\) where \(\omega > 0\).
    \[\Theta''(\theta) = -\omega^2 \Theta(\theta) \implies \Theta(\theta) = A\cos(\omega\theta) + B\sin(\omega\theta).\]
    Using the boundary conditions:
    \[\Theta(0) = A = 0, \quad \Theta\left(\frac{\pi}{2}\right) = B\sin\left(\frac{\omega\pi}{2}\right) = 0.\]
    Then \(\omega\frac{\pi}{2} = n\pi\) for some \(n \in \Z\), and we find that \(\omega = 2n\). Thus, \(\Theta(\theta) = b_n\sin(2n\theta)\) for \(n = 1, 2, 3, \ldots\) and \(\lambda = 4n^2\).

    Now we solve the ODE in \(r\).
    \[\frac{r^2R'' + rR'}{R} = 4n^2 \implies r^2R'' + rR' - 4n^2R = 0, \quad n = 1, 2, 3, \ldots\]
    We recognize this as a Cauchy-Euler equation, and we make the substitution \(R(r) = r^\alpha\).
    \begin{align*}
        r^2\alpha(\alpha - 1)r^{\alpha - 2} + r\alpha r^{\alpha - 1} - 4n^2r^\alpha & = 0                           \\
        r^\alpha(\alpha^2 - \alpha + \alpha - 4n^2)                                 & = 0                           \\
        \alpha^2 - 4n^2                                                             & = 0 \implies \alpha = \pm 2n.
    \end{align*}
    This gives us two linearly independent solutions \(r^{2n}\) and \(r^{-2n}\), so the general solution is
    \[R(r) = Cr^{-2n} + Dr^{2n}, \quad n = 1, 2, 3, \ldots\]
    However, we must have that the solution is bounded at \(r = 0\), so we must have \(C = 0\).

    The general solution for \(u(r, \theta)\) is
    \[u(r, \theta) = \sum_{n = 1}^\infty b_n r^{2n}\sin(2n\theta).\]

    The boundary condition \(u(1, \theta) = f(\theta)\) gives
    \[f(\theta) = \sum_{n = 1}^\infty b_n\sin(2n\theta).\]

    We wish to find the Fourier coefficients \(b_n\). If we multiply both sides by \(\sin(2k\theta)\) and integrate from \(0\) to \(\pi/2\), we find that
    \[\int_0^{\pi/2} f(\theta)\sin(2k\theta) \dd{\theta} = \sum_{n = 1}^\infty b_n\int_0^{\pi/2} \sin(2n\theta)\sin(2k\theta) \dd{\theta}.\]
    We can rewrite the integral on the right-hand side to be from \(0\) to \(\pi\).
    \[\int_0^{\pi/2} f(\theta)\sin(2k\theta) \dd{\theta} = \sum_{n = 1}^\infty b_n\int_0^{\pi} \sin(n\theta)\sin(k\theta) \dd{\theta}.\]
    Now we can recall that for the integral on the right hand side
    \[\int_0^\pi \sin(n\theta)\sin(k\theta) \dd{\theta} = \langle \sin(n\theta), \sin(k\theta) \rangle_{(0,\,\pi)} = \begin{cases} 0 & n \neq k \\ \pi/2 & n = k \end{cases}.\]
    So we find that
    \[b_n = \frac{2}{\pi}\int_0^{\pi/2} f(\theta)\sin(2n\theta) \dd{\theta}.\]
    and the solution is
    \[u(r, \theta) = \frac{2}{\pi}\sum_{n = 1}^\infty \left[\int_0^{\pi/2} f(\theta)\sin(2n\theta) \dd{\theta}\right] r^{2n}\sin(2n\theta).\]
\end{solution}

\pagebreak

\begin{problem}
Find the solution to the following boundary value problem on an annulus:
\[
    \begin{cases}
        \begin{aligned}
            \Delta u     & = 0,            &  & 1 < r < 2, \quad 0 < \theta < 2\pi, \\
            u(1, \theta) & = \sin(\theta), &  & 0 < \theta < 2\pi,                  \\
            u(2, \theta) & = \cos(\theta), &  & 0 < \theta < 2\pi.
        \end{aligned}
    \end{cases}
\]
\begin{hint}
    \emph{Since we are working on an annulus, we do not need boundedness condition for \(u\) at \(r = 0\).}
\end{hint}
\end{problem}
\begin{solution}
    As in the previous problem, we arrive at
    \[\frac{r^2R'' + rR'}{R} = -\frac{\Theta''}{\Theta} = \lambda.\]
    We solve the ODE in \(\theta\) first with the implicit periodic boundary conditions. We recall from lecture that the general solution is
    \[\Theta(\theta) = A_n\cos(n\theta) + B_n\sin(n\theta), \quad n = 0, 1, 2, \ldots, \quad \lambda = n^2.\]

    We also recall that the general solution for \(R(r)\) given \(\lambda = n^2\) is
    \[R =
        \begin{cases}
            \begin{aligned}
                 & C_n r^{-n} + D_nr^n, &  & n = 1, 2, 3, \ldots, \\
                 & C_0\ln(r) + D,       &  & n = 0.
            \end{aligned}
        \end{cases}
    \]
    Here, we do not set \(C_n = 0\) or \(C_0 = 0\) as we do not need to enforce boundedness at \(r = 0\) since the annulus is bounded by \(1 < r < 2\).

    The general solution for \(u(r, \theta)\) is
    \[u(r, \theta) = A + B\ln(r) + \sum_{n = 1}^\infty \left[A_n\cos(n\theta) + B_n\sin(n\theta)\right]\left[C_nr^{-n} + D_nr^n\right]\]

    Using the first boundary condition, we have:
    \[u(1, \theta) = A + \sum_{n = 1}^\infty \left[A_n\cos(n\theta) + B_n\sin(n\theta)\right]\left[C_n + D_n\right] = \sin(\theta).\]
    We recognize this as a Fourier series, and we can compute the Fourier coefficients.
    \[A = \int_0^{2\pi} \sin(\theta) \dd{\theta} = 0\]
    \[A_n = \frac{1}{\pi}\int_0^{2\pi} \sin(\theta)\cos(n\theta) \dd{\theta} = 0\]
    \[B_n = \frac{1}{\pi}\int_0^{2\pi} \sin(\theta)\sin(n\theta) \dd{\theta} = \begin{cases} \frac{1}{C_1 + D_1} & n = 1 \\ 0 & n \neq 1 \end{cases}\]

    Now we have
    \[u(r, \theta) = B\ln(r) + \frac{1}{C_1 + D_1}\sin(\theta)\left[C_1 + D_1\right] = B\ln(r) + \sin(\theta)\]

    Using the second boundary condition, we have:
    \[u(2, \theta) = B\ln(2) + \sin(\theta) = \cos(\theta) \implies B = \frac{\cos(\theta) - \sin(\theta)}{\ln(2)}\]

    The solution is
    \[u(r, \theta) = \frac{\ln(r)}{\ln(2)}\bigl(\cos(\theta) - \sin(\theta)\bigr) + \sin(\theta).\]

    % Graph: https://www.desmos.com/3d/5c143829fb
\end{solution}

\pagebreak

\begin{problem}
Consider the following harmonic function in a unit disk with boundary value:
\[
    \begin{cases}
        \begin{aligned}
            \Delta u     & = 0,                  &  & 0 < r < 1, \quad 0 < \theta < 2\pi, \\
            u(1, \theta) & = 1 + 3\sin(2\theta), &  & 0 < \theta < 2\pi.
        \end{aligned}
    \end{cases}
\]
\textbf{Without} solving the solution explicitly, answer the following questions:
\begin{enumerate}[(a)]
    \item What is the maximum value of \(u\) in the unit disk \(\{x^2 + y^2 \leq 1\}\)?
    \item What is the value of \(u\) at the origin?
\end{enumerate}
\end{problem}
\begin{solution}
    \leavevmode
    \begin{enumerate}[(a)]
        \item By the maximum principle, the maximum value of \(u\) in the unit disk is the maximum value of \(u\) on the boundary. The maximum value of \(u\) on the boundary is \(1 + 3\), so the maximum value of \(u\) in the unit disk is \(4\).
        \item The origin is the center of the disk, so the value of \(u\) at the origin is the average at the boundary by the mean value property.
              \begin{align*}
                   & \frac{1}{2\pi}\int_0^{2\pi} \left(1 + 3\sin(2\theta)\right) \dd{\theta}                            \\
                   & = \frac{1}{2pi}\int_0^{2\pi} 1 \dd{\theta} + \frac{3}{2\pi}\int_0^{2\pi} \sin(2\theta) \dd{\theta} \\
                   & = 1 + 0                                                                                            \\
                   & = 1.
              \end{align*}
    \end{enumerate}
\end{solution}

\pagebreak

\begin{problem}
Let \(\Omega\) be a bounded connected open set. The goal of this problem is to show the uniqueness of harmonic function with Dirichlet boundary condition by an energy method.
\begin{enumerate}[(a)]
    \item Recall that for any vector field \(\overrightarrow{A} = (A_1, A_2)\), \(\Div{\overrightarrow{A}} = A_{1x} + A_{2y}\), and for any scalar function \(f\), \(\nabla f = (f_x, f_y)\). Verify that
          \[
              \Div{(u\nabla u)} = u_x^2 + u_y^2 + u\Delta u.
          \]
    \item Recall the divergence theorem from multivariable calculus:
          \[\int_\Omega \Div{\overrightarrow{A}} \dd{x} \dd{y} = \int_{\partial \Omega} \overrightarrow{A} \cdot n \dd{S},\]
          where \(n\) is the unit outer normal vector on \(\partial \Omega\). Use the divergence theorem with \(\overrightarrow{A} = u\nabla u\) to show that the only function satisfying
          \[
              \begin{cases}
                  \begin{aligned}
                      \Delta u & = 0, &  & \text{in } \Omega,         \\
                      u        & = 0  &  & \text{on } \partial \Omega
                  \end{aligned}
              \end{cases}
          \]
          is \(0\).
\end{enumerate}
\end{problem}
\begin{solution}
    \leavevmode
    \begin{enumerate}[(a)]
        \item \leavevmode
              \begin{align*}
                  \Div{(u\nabla u)} & = \Div{\left(u\left(u_x, u_y\right)\right)}                            \\
                                    & = \Div{\left(uu_x, uu_y\right)}                           \\
                                    & = (uu_x)_x + (uu_y)_y = u_x^2 + uu_{xx} + u_y^2 + uu_{yy} \\
                                    & = u_x^2 + u_y^2 + u \left(u_{xx} + u_{yy}\right)         \\
                                    & = u_x^2 + u_y^2 + u\Delta u.
              \end{align*}
        \item By the divergence theorem,
              \[\int_\Omega \Div{(u\nabla u)} \dd{x} \dd{y} = \int_{\partial \Omega} u\nabla u \cdot n \dd{S}.\]
              Since \(u = 0\) on \(\partial \Omega\), we have
              \[\int_\Omega \Div{(u\nabla u)} \dd{x} \dd{y} = 0.\]
              By part (a), we have
              \[\int_\Omega \left(u_x^2 + u_y^2 + u\Delta u\right) \dd{x} \dd{y} = 0.\]
              Since \(u\) is harmonic, \(\Delta u = 0\), so we have
              \[\int_\Omega \left(u_x^2 + u_y^2\right) \dd{x} \dd{y} = 0.\]
              Since \(u_x^2 + u_y^2 \geq 0\), we must have that \(u_x^2 + u_y^2 = 0\), so \(u_x = u_y = 0\). Thus, \(u\) is constant, and since \(u = 0\) on \(\partial \Omega\), we must have that \(u = 0\) in \(\Omega\).

    \end{enumerate}
\end{solution}



\end{document}
